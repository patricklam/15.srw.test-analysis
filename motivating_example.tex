The goal of our analysis is to report clone sets which are candidates
for refactoring by the developer.  
Our primary
assumption is that assertions are frequently a key part of test
methods. Therefore, assertion characteristics
help us understand the structure and
semantics of test methods, and we can use that understanding to detect clones.

Intuitively, our analysis aims to declare that $m_1$ is a clone of
method $m_2$ if they are similar in terms of control flow and
assertion/fail calls.

Consider, as an example, Figure~\ref{fig:motivating_code}, from a clone set reported by our
analysis. The full set contains 4 similar methods from Weka's test
suite.  Manual inspection confirms that these methods are
indeed clones.  Figure~\ref{fig:motivating_refactored}
shows one plausible refactoring of the code from
Figure~\ref{fig:motivating_code}. While, after refactoring, individual tests become
more complicated to understand, we argue that the suite of 4 refactored tests 
is collectively easier to understand and maintain.

Our analysis identifies the clone set in Figure~\ref{fig:motivating_code} by
matching fingerprints from the assertion invocations in the methods. 
In this case, all four test methods only have one assertion invocation, {\tt assertTrue(boolean)}. 

We next compute assertion fingerprints for each of the assertions.  
Assertion fingerprints consist of 5 components; we illustrate 
branch counts and merge counts here. Each assertion is reachable
from its method start through a single branch, \predicate{i <
result.numAttributes()}, and thus has a branch count of 1. Furthermore,
there is an implicit branch and merge in the boolean comparison in the
assertion parameter (\predicate{.type() != ...}), which leads to a merge
count of 1 for each assertion. Finally, these assertions are not in loops,
catch blocks, or caught by exception handlers. This gives assertion
fingerprints \predicate{(bc:1, mc:1)} for all asserts.

Because all four of the test methods share the same ordered set of
assertion fingerprints, and no other methods have the same
fingerprints, our technique reports that the test methods in
Figure~\ref{fig:motivating_code} make up a clone set.

\thesis{
%setID: 37
\begin{figure}[ht!]
\begin{subfigure}[b]{\columnwidth}
\begin{lstlisting}
public void testNominalFiltering() {
   m_Filter = getFilter(Attribute.NOMINAL);
   Instances result = useFilter();
   for (int i = 0; i < result.numAttributes(); i++)
      assertTrue(result.attribute(i).type() != Attribute.NOMINAL);
}
\end{lstlisting}
\end{subfigure}

\begin{subfigure}[b]{\columnwidth}
\begin{lstlisting}
public void testStringFiltering() {
   m_Filter = getFilter(Attribute.STRING);
   Instances result = useFilter();
   for (int i = 0; i < result.numAttributes(); i++)
      assertTrue(result.attribute(i).type() != Attribute.STRING);
}
\end{lstlisting}
\end{subfigure}

\begin{subfigure}[b]{\columnwidth}
\begin{lstlisting}
public void testNumericFiltering() {
   m_Filter = getFilter(Attribute.NUMERIC);
   Instances result = useFilter();
   for (int i = 0; i < result.numAttributes(); i++)
      assertTrue(result.attribute(i).type() != Attribute.NUMERIC);
}
\end{lstlisting}
\end{subfigure}

\begin{subfigure}[b]{\columnwidth}
\begin{lstlisting}
public void testDateFiltering() {
   m_Filter = getFilter(Attribute.DATE);
   Instances result = useFilter();
   for (int i = 0; i < result.numAttributes(); i++)
      assertTrue(result.attribute(i).type() != Attribute.DATE);
}
\end{lstlisting}
\end{subfigure}

   \hrule
   \caption
   {\label{fig:motivating_code}
   A clone set of four refactorable test method clones reported from Weka's test suite
   ({\tt weka.filters.unsupervised.attribute.RemoveTypeTest}).}
\end{figure}


}
\paper{
%setID: 37
\begin{lstlisting}
public void testNominalFiltering() {
   m_Filter = getFilter(Attribute.NOMINAL);
   Instances result = useFilter();
   for (int i = 0; i < result.numAttributes(); i++)
      assertTrue(result.attribute(i).type() != Attribute.NOMINAL);
}
\end{lstlisting}
\begin{lstlisting}
public void testStringFiltering() {
   m_Filter = getFilter(Attribute.STRING);
   Instances result = useFilter();
   for (int i = 0; i < result.numAttributes(); i++)
      assertTrue(result.attribute(i).type() != Attribute.STRING);
}
\end{lstlisting}


}

\begin{figure}[ht!]
   \centering
\thesis{
   \begin{tikzpicture}[
         >=latex, 
         node distance=4cm
      ]
}
\paper{
   \begin{tikzpicture}[
         >=latex, 
         node distance=3cm
      ]
}
\slides{
   \begin{tikzpicture}[
         >=latex, 
         node distance=2.5cm
      ]
      \scriptsize
}

      \node (0) [initial, block] {\tt // ...}; 
      \node (1) [branch, below of=0, yshift=3.5em] {\predicate{i < result.numAttributes()}};
      \node (2) [branch, below of=1] {
         {\tt int type;}\\
         {\tt /* type in }\\
         {\tt * \{ Attribute.NOMINAL,}
         {\tt Attribute.STRING,}\\
         {\tt * ~~Attribute.NUMERIC,}
         {\tt Attribute.DATE}
         {\tt \}}\\
         {\tt */}\\
         \predicate{result.attribute(i).type() != type}
      }; 
      \node (3) [block, accepting, right of=1, xshift=3em] {}; 

      \node (4) [merge, below of=2] {\tt assertTrue(boolean)}; 

      \path[->]

      (0) edge[thick] (1)

      (1) edge[thick] node [left] {true} (2)
          edge[thick] node [above] {false} (3)

      (2) edge[thick, bend right=15] node [left] {true} (4)
          edge[thick, bend left=15] node [right] {false} (4)

      (4) edge[thick, bend right=85] (1)
      ;

   \end{tikzpicture}
   \vspace{2pt}
   \notslides{
   \hrule
   \caption{\label{fig:motivating_graph}
      A control-flow graph for the example in Figure~\ref{fig:motivating_code}. 
      Only assertions and predicates included.
   }
   }
\end{figure}


\begin{lstlisting}
  static final int [] filteringTypes = {
    Attribute.NOMINAL, Attribute.STRING,
    Attribute.NUMERIC, Attribute.DATE
  };

  public void testFiltering() {
    for (int type : filteringTypes)
      testFiltering(type);
  }

  public void testFiltering(final int type) {
    m_Filter = getFilter(type);
    Instances result = useFilter();
    for (int i = 0; i < result.numAttributes(); i++)
      assertTrue(result.attribute(i).type() != type);
  }
\end{lstlisting}

