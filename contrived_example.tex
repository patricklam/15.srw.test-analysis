The code in Figure~\ref{fig:contrived_code}, and its CFG in
Figure~\ref{fig:contrived_graph}, 
illustrate all five components of a fingerprint. For expository reasons, we have included fingerprints for each vertex $v$ in Figure~\ref{fig:contrived_graph} (even though only fingerprints for assertion invocations matter for our technique).

Observe that
%{\tt assertEquals(int, int)} (line 5) has a branch count of 2 (\predicate{i < 10} and \predicate{i == 2}); 
%{\tt assertTrue(boolean)} (line 6) has a branch count of 1 (\predicate{i < 10}) and merge count of 2 (\predicate{i == 2} and \predicate{i != 10});
%{\tt fail(String)} (line 10) has a branch count of 1 ($\neg$\predicate{i < 10}) and 1 exceptional successor (one corresponding catch block); 
%{\tt assertEquals(int, int)} (line 12) has a branch count of 1 ($\neg$\predicate{i < 10}), a merge count of 1 (\predicate{i > 10}), and is in a catch block.
%The calls {\tt assertEquals(int, int)} (line 5) and {\tt
%assertTrue(boolean)} (line 6) are in a loop (for loop at line 3).
the branch vertex at line 3 (\predicate{i < 10}) increments the branch count for its successors at line 4 and 5 (inside the for loop) and those at line 10 and 12 (outside the for loop).
Line 6 has two merges: one from \predicate{i == 2} and the other from \predicate{i != 10}.
Although line 10 ({\tt fail(String)}) is actually unreachable and hence does not throw any exceptions, Soot includes its exceptional successors anyway.
%line 5 and 6 being in loops and line 12 being in a catch block for obvious reasons.

\begin{lstlisting}
public void test() {
   int i;
   for (i = 0; i < 10; ++i) {
      if (i == 2)
         assertEquals(i, 2);
      assertTrue(i != 10);
   }
   try {
      throw new Exception();
      fail("Should have thrown exception");
   } catch (final Exception e) {
      assertEquals(i, 10);
   }
}
\end{lstlisting}

   \begin{tikzpicture}[
         >=latex, 
         node distance=2cm
      ]
      \tiny

      \node (0) [initial, branch] {
         {\tt /*}
         \\{\tt * line 3}
         \\{\tt */}
         \\\predicate{i < 10}
      }; 
      \node (1) [branch, right of=0, xshift=1.8cm] {
         {\tt /*}
         \\{\tt * line 4}
         \\{\tt * bc:1 (\predicate{i<10})}
         \\{\tt * inLoop:true}
         \\{\tt */}
         \\\predicate{i == 2}
      }; 
      \node (2) [try, below of=0, yshift=-1cm] {
         {\tt try:}
         \\{\tt /*}
         \\{\tt * line 10}
         \\{\tt * bc:1 ($\neg$\predicate{i<10})}
         \\{\tt * es:1}
         \\{\tt */}
         \\{\tt fail(String)}
      }; 

      \node (3) [block, below of=1] {
         {\tt /*}
         \\{\tt * line 5}
         \\{\tt * bc:2 (\predicate{i<10},\predicate{i==2})}
         \\{\tt * inLoop:true}
         \\{\tt */}
         \\{\tt assertEquals(int, int)}
      }; 
      \node (4) [merge, below of=3, yshift=-.5cm] {
         {\tt /*}
         \\{\tt * line 6}
         \\{\tt * bc:1 (\predicate{i<10})}
         \\{\tt * mc:1 (\predicate{i==2})}
         \\{\tt * inLoop:true}
         \\{\tt */}
         \\\predicate{i != 10}
      };

      \node (5) [catch, below of=2, yshift=-1cm] {
         {\tt catch:}
         \\{\tt /*}
         \\{\tt * line 12}
         \\{\tt * bc:1 ($\neg$\predicate{i<10})}
         \\{\tt * inCatch:true}
         \\{\tt */}
         \\{\tt assertEquals(int, int)} }; 

      \node (7) [merge, below of=4, yshift=-1cm] {
         {\tt /*}
         \\{\tt * line 6}
         \\{\tt * bc:1 (\predicate{i<10})}
         \\{\tt * mc:2 (\predicate{i==2},\predicate{i!=10})}
         \\{\tt * inLoop:true}
         \\{\tt */}
         \\{\tt assertTrue(boolean)}
      };

      \node (6) [accepting, block, below of=5, yshift=-.3cm] {}; 

      \path[->]

      (0) edge[thick] node [above] {true} (1)
          edge[thick] node [right] {false} (2)

      (1) edge[thick] node [left] {true} (3)
          edge[thick, bend left=47] node [right] {false} (4)

      (2) edge[thick, dashed] node [right] {Exception} (5)
          edge[thick, bend right=47] (6)

      (3) edge[thick] (4)

      (4) edge[thick, bend right=10] node [left] {true} (7)
          edge[thick, bend left=10] node [right] {false} (7)

      (7) edge[thick, out=130, in=320] (0)

      (5) edge[thick] (6)
      ;

   \end{tikzpicture}

